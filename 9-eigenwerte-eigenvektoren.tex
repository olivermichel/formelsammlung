%!TEX root = formelsammlung-master.tex

\section{Eigenwerte und Eigenvektoren}
\label{sec:eigenwerte_und_eigenvektoren}

\subsection{Definition}
\label{sub:definition}
Jede $n \times n$ - Matrix beschreibt eine lineare Abbildung im $\mathbb{R}^n$.
Als Eigenvektor einer solchen linearen Abbildung/Matrix bezeichnet man einen Vektor, der durch die jeweilige 
Abbildung in seiner Richtung nicht verändert wird.\\
Lediglich Streckungen und Stauchungen sind möglich. Den Faktor der entsprechenden 
Streckung oder Stauchung ist dann der passende Eigenwert zum Eigenvektor.\\
Der Null-Vektor ist trivialerweise natürlich auch ein Eigenvektor jeder lin. Abbildung, wird jedoch nicht beachtet.
