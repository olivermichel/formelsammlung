%!TEX root = formelsammlung-master.tex
\section{Numerik}
\label{sec:numerik}

\subsection{Fixpunkte}
\label{sub:fixpunkte}

Sei $f : \mathbb{R} \rightarrow \mathbb{R}$ gegeben, dann heißen die Lösungen der Gleichung 
\begin{equation}
	y = f(x) = x
\end{equation}
\emph{Fixpunkte} von $f$.

\subsection{\"{A}quivalenz zwischen Fixpunkt- und Nullstellenproblem}
\label{sub:aequivalenz_zwischen_fixpunkt_und_nullstellenproblem}

Jedes Fixpunktproblem $f(x) = x$ lässt sich in das äquivalente Nullstellenproblem $f(x) = 0$ umwandeln.
\begin{equation}
	f(x) = x \Leftrightarrow f(x) - x = 0
\end{equation}
Umgekehrt ergibt sich aus jedem Nullstellenproblem $f(x) = 0$ ein entsprechendes Fixpunktproblem.
\begin{equation}
	f(x) = 0 \Leftrightarrow f(x) + x = x
\end{equation}

\subsection{Newton-Raphson Iteration zur Nullstellenbestimmung}
\label{sub:newton_raphson_iteration_zur_nullstellenbestimmung}

Mit einem gewählten Startwert $x_0$ lässt sich durch die Newton-Raphson-Iteration eine 
Nullstelle einer Funktion $f(x)$ numerisch berechnen. Die Iteration hierzu sieht 
folgendermaßen aus:
\begin{equation}
	x_{n+1} = x_n - \frac{f(x_n)}{f'(x_n)}
\end{equation}
Bei einem geeigneten Startwert läuft die Iteration gegen eine Nullstelle von $f(x)$.