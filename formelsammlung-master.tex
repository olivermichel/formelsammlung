%
%  Formel und Verfahrensammlung
%
%  Created by Oliver Michel on 2009-11-18.
%  Copyright (c) 2009 :editum.internet solutions. All rights reserved.
%
\documentclass[a4paper,11pt]{article}
\usepackage[utf8]{inputenc}
\usepackage{fullpage, listings, ngerman, graphicx, hyperref, fancyhdr, multicol, amssymb, setspace}

\title{Formel- und Verfahrensammlung Analysis}
\author{Oliver Michel\\Universität Wien}
\date{\today}

\onehalfspacing

\begin{document}
		
	\maketitle
	\tableofcontents

	\pagebreak

	\section{Grundlagen} % (fold)
	\label{sec:grundlagen}
	
	\subsection{Ableitungsregeln} % (fold)
	\label{sub:ableitungsregeln}
	
	\subsubsection{Faktorregel}
	\label{ssub:faktorregel}
	\begin{equation}
		(ax^n)' = a(x^n)' 
	\end{equation}	
	
	\subsubsection{Summenregel}
	\label{ssub:summenregel}
	\begin{equation}
		(u+v)' = u' + v' 
	\end{equation}	
	
	\subsubsection{Potenzregel}
	\label{ssub:potenzregel}
	\begin{equation}
		(x^n)' = nx^{n-1} 
	\end{equation}

	\subsubsection{Produktregel}
	\label{ssub:produktregel}
	\begin{equation}
		(uv)' = u'v + uv'
	\end{equation}	

	\subsubsection{Quotientenregel}
	\label{ssub:quotientenregel}
	\begin{equation}
		\left(\frac{u}{v}\right)' = \frac{u'v - uv'}{v^2}
	\end{equation}
	
	\subsubsection{Kettenregel}
	\label{ssub:kettenregel}
	\begin{equation}
		(u \circ v)' = u'(v) \cdot v'
	\end{equation}
	Beispiel: $(sin(x^2))' = cos(x^2) \cdot 2x$
	% subsection ableitungsregeln (end)
	
	\subsection{Besondere Ableitungen} % (fold)
	\label{sub:besondere_ableitungen}
	
	\subsubsection{Winkelfunktionen}
	\label{ssub:winkelfunktionen}	
	\begin{eqnarray}
		sin' &=& cos \\
		cos' &=& -sin \\
		(-sin)' &=& -cos \\
		(-cos)' &=& sin
	\end{eqnarray}
	
	\subsubsection{Exponentialfunktionen}
	\label{ssub:exponentialfunktionen}	
	\begin{eqnarray}
		(e^x)' &=& e^x \\
		(e^{kx})' &=& k \cdot e^{kx} \\
		(a^x)' &=& ln(a) \cdot a^x
	\end{eqnarray}
	
	\subsubsection{Logarithmusfunktionen}
	\label{ssub:exponentialfunktionen}	
	\begin{eqnarray}
		(ln(x))' &=& \frac{1}{x} \\
		(log_b(x))' &=& \frac{1}{x\cdotln(b)}
	\end{eqnarray}	
	
	% subsection besondere_ableitungen (end)
	
	\subsection{Integration} % (fold)
	\label{sub:integration}
	
	\subsubsection{Hauptsatz der Integralrechnung} 
	\label{ssub:hauptsatz_der_integralrechnung}
	
	\begin{equation}
		\int_a^b f(x)dx = F(b) - F(a)
	\end{equation}
	
	\subsubsection{Integration der Exponentialfunktion}
	\label{ssub:integration_der_exponentialfunktion}
	
	Wenn der Exponent der Exponentialfunktion eine lineare Funktion der Form $mx+c$ ist, dann lautet die Stammfunktion
	\begin{equation}
		f(x) = e^{mx+c} \rightarrow F(x) = \frac{1}{m} e^{mx+c}
	\end{equation}
	
	\subsubsection{Partielle Integration}
	\label{ssub:partielle_integration}
	\begin{equation}
		\int_a^b (u \cdot v') dx = \left[u \cdot v \right]^a_b - \int_a^b (u' \cdot v) dx
	\end{equation}
	% subsection integration (end)
	
	\subsection{Vektoren} 
	\label{sub:vektoren}
	
	\subsubsection{Addition und Substraktion von Vektoren} % (fold)
	\label{ssub:addition_und_substraktion_von_vektoren}
	
	\begin{eqnarray}
		\overrightarrow{a} + \overrightarrow{b}
		= \left( \begin{array}{c} a_1\\a_2\\a_3\end{array}\right) + \left( \begin{array}{c} b_1\\b_2\\b_3\end{array}\right)
		= \left( \begin{array}{c} a_1 + b_1\\a_2 + b_2\\a_3 + b_3\end{array}\right) \\
		\overrightarrow{a} - \overrightarrow{b}
		= \left( \begin{array}{c} a_1\\a_2\\a_3\end{array}\right) - \left( \begin{array}{c} b_1\\b_2\\b_3\end{array}\right)
		= \left( \begin{array}{c} a_1 - b_1\\a_2 - b_2\\a_3 - b_3\end{array}\right)
	\end{eqnarray}
	
	\subsubsection{Skalarmultiplikation}
	\label{ssub:skalarmultiplikation}
	
	\begin{equation}
		r \cdot \overrightarrow{v} = r \left( \begin{array}{c} v_1\\v_2\\v_3\end{array}\right)
		= \left( \begin{array}{c}r \cdot v_1\\r \cdot v_2\\r \cdot v_3\end{array}\right)
	\end{equation}

	\subsubsection{Skalarprodukt (Vektorprodukt)} % (fold)
	\label{ssub:skalarprodukt_vektorprodukt_}
	
	\begin{equation}
		\overrightarrow{a} \cdot \overrightarrow{b}
		= \left( \begin{array}{c} a_1\\a_2\\a_3\end{array}\right) \cdot \left( \begin{array}{c} b_1\\b_2\\b_3\end{array}\right)
		= a_1 \cdot b_1 + a_2 \cdot b_2 + a_3 \cdot b_3
	\end{equation}
	
	\subsubsection{Betrag eines Vektors}
	\label{ssub:betrag_eines_vektors}
	
	\begin{equation}
		|\overrightarrow{v}| = \sqrt{v_1^2 + v_2^2 + v_3^2}
	\end{equation}
	
	% subsection vektoren (end)
	
	\subsection{Matrizzen} % (fold)
	\label{sub:matrizzen}
	
	\subsubsection{Matrixmultiplikation} 
	\label{ssub:matrixmultiplikation}
	
	Bei der Multiplikation von 2 Matrizzen muss die Zeilenanzahl der ersten Matrix gleich der Spaltenanzahl der 2. Matrix sein. 
	\begin{equation}
		C = A \cdot B, A : l \times m, B : m \times n \rightarrow C : l \times n
	\end{equation}
	Beispiel: 
	\begin{equation}
		\left(\begin{array}{ccc}1 & 2 & 3 \\4 & 5 & 6 \\\end{array}\right) \cdot
		\left(\begin{array}{cc}6 & -1 \\3 & 2 \\0 & -3\end{array}\right)
		=\left(\begin{array}{cc}1 \cdot 6  +  2 \cdot 3  +  3 \cdot 0 &
		  1 \cdot (-1) +  2 \cdot 2 +  3 \cdot (-3) \\4 \cdot 6  +  5 \cdot 3  +  6 \cdot 0 &
		  4 \cdot (-1) +  5 \cdot 2 +  6 \cdot (-3) \\\end{array}\right)
		=\left(\begin{array}{cc}12 & -6 \\39 & -12\end{array}\right)
	\end{equation}
	
	\subsubsection{Multiplikation eines Vektors mit einer Matrix}
	\label{ssub:multiplikation_eines_vektors_mit_einer_matrix}

	Die Multiplikation eines Vektors mit einer Matrix ergibt immer einen Vektor. Die Komponenten des Vektors ergeben sich
	aus der Multiplikation der jeweiligen Zeile der Matrix mit dem gesamten Vektor (Skalarprodukt).
	\\Beispiel:
	\begin{equation}
		\left(\begin{array}{c}1\\2\end{array}\right) \cdot \left(\begin{array}{cc}3 & 4\\5 & 6\end{array}\right)
		= \left(\begin{array}{c}1 \cdot 3 + 2 \cdot 4 \\ 1 \cdot 5 + 2 \cdot 6\end{array}\right)
		= \left(\begin{array}{c}11\\17\end{array}\right)
	\end{equation}
	
	\subsubsection{Kriterium von Sylvester}
	\label{ssub:kriterium_von_sylvester}
	
	Eine symmetrische Matrix $A$ ist 
	\begin{itemize}
		\item positiv definit, wenn alle ihre Hauptminoren\footnote{Hauptminoren: Determinanten von Untermatrizzen} positiv sind.
		\item negativ definit, wenn ihre Hauptminoren alternierend negativ und positiv sind.
		\item indefinit, wenn keine der beiden oberen Kriterien zutrifft.
	\end{itemize}
	
	% subsection matrizzen (end)
	
	\subsection{Kreise} % (fold)
	\label{sub:kreise}
	
	\subsubsection{Kreisgleichung} 
	\label{ssub:kreisgleichung}
	
	Ein Kreis mit dem Radius $r$ um den Mittelpunkt $M$ ist definiert durch
	\begin{equation}
		y = y_M \pm \sqrt{r^2 - (x-x_M)^2}
	\end{equation}
	Ist der Mittelpunkt der Ursprung des Koordinatensystems, fällt der Punkt M weg:
	\begin{equation}
		y = \pm \sqrt{r^2-x^2}
	\end{equation}
	Handelt es sich um einen Einheitskreis um den Ursprung lautet die Gleichung also
	\begin{equation}
		y = \pm \sqrt{1-x^2}
	\end{equation}
	% subsection kreise (end)
	% section grundlagen (end)
	
	\pagebreak
	
	\section{Mehrdimensionale Funktionen} % (fold)
	\label{sec:mehrdimensionale_funktionen}
	\subsection{Abbildungstypen} % (fold)
	\label{sub:abbildungstypen}
	
	\subsubsection{$\mathbb{R}$ \rightarrow \mathbb{R}}
	\label{ssub:r_rightarrow_r}
	Beispiel mit 1. Ableitung:
	\begin{eqnarray*}
		f(x) &=& sin(x)cos(x) \\
		f'(x) &=& cos^2(x)-sin^2(x)
	\end{eqnarray*}

	\subsubsection{$\mathbb{R}$ \rightarrow \mathbb{R}^n}
	\label{ssub:r_rightarrow_r}
	Beispiel mit 1. Ableitung (Ableitungsvektor):
	\begin{eqnarray*}
		f(t) &=& \left(\begin{array}{c} cos(t)) \\ sin(t))\end{array}\right) \\
		\nabla f(t) &=& \left( \begin{array}{c} -sin(t) \\ cos(t)\end{array} \right)
	\end{eqnarray*}

	\subsubsection{$\mathbb{R}^n$ \rightarrow \mathbb{R}}
	\label{ssub:r_rightarrow_r}
	Beispiel mit 1. Ableitung (Gradientenvektor):
	\begin{eqnarray*}
		f(x_1,x_2) &=& sin(x_1)cos(x_2) \\
		\nabla f(x_1,x_2) &=& \left( \begin{array}{c} cos(x_1) cos(x_2) \\ -sin(x_1)sin(x_2) \end{array} \right)
	\end{eqnarray*}
		
	\subsubsection{$\mathbb{R}^n$ \rightarrow \mathbb{R}^m}
	\label{ssub:r_rightarrow_r}
	Beispiel mit 1. Ableitung (Jacobi-Matrix): 
	\begin{eqnarray*}
		f(x_1,x_2) &=& \left(\begin{array}{c} sin(x_1) \\ cos(x_2)\end{array}\right) \\
		\nabla f(x_1,x_2) = J_f &=& \left( \begin{array}{cc} cos(x_1) & 0 \\ 0 & -sin(x_2) \end{array} \right)
	\end{eqnarray*}	
	% subsection abbildungstypen (end)
	
	\subsection{Niveaulinien} % (fold)
	\label{sub:niveaulinien}
	
	\begin{equation}
		f(x_1,x_2,...,x_n) = y = c
	\end{equation}
	Auflösen nach $x_2$ und Niveauebenen für $c$ einsetzen.
	
	% subsection niveaulinien (end)
	
	% section funktionen_mehrerer_variablen (end)
	
	\pagebreak
	
	\section{Differentialrechnung bei Funktionen mehrerer Variablen} % (fold)
	\label{sec:differentialrechnung_bei_funktionen_mehrerer_variablen}
	
	\subsection{Partielles Differenzieren} % (fold)
	\label{sub:partielles_differenzieren}

	Partielle Ableitung von $f$ nach $x_1$: $\frac{\partial f}{\partial x_1} f(x_1,x_2,...,x_n)$
	% subsection partielles_differenzieren (end)
	
	\subsection{Gradientenvektor} % (fold)
	\label{sub:gradientenvektor}
	Gradientenvektor $\nabla f$ zu $f(x_1,x_2,...,x_n)$: 
	\begin{equation}
		grad(f) = \nabla f = \left(\begin{array}{c} \frac{\partial f}{\partial x_1} \\ 
		\frac{\partial f}{\partial x_2} \\ \vdots \\ \frac{\partial f}{\partial x_n} \end{array}\right)
	\end{equation}
	Vektor aus partiellen Ableitungen nach allen Argumenten von $f$
	% subsection gradientenvektor (end)
	
	\subsection{Hesse-Matrix} % (fold)
	\label{sub:hesse_matrix}
	Hesse-Matrix $H_f$ zu $f(x_1,x_2,...,x_n)$: 
	
	\begin{equation}
		H_f = \nabla^2f = \left( \begin{array}{cccc}
			\frac{\partial^2 f}{\partial x_1^\partial x_1} & \frac{\partial^2 f}{\partial x_1^\partial x_2} 
			& \dots & \frac{\partial^2 f}{\partial x_1^\partial x_n} \\
			
			\frac{\partial^2 f}{\partial x_2^\partial x_1} & \frac{\partial^2 f}{\partial x_2^\partial x_2} 
			& \dots & \frac{\partial^2 f}{\partial x_2^\partial x_n} \\	
			\vdots & \vdots & \ddots & \vdots \\
			\frac{\partial^2 f}{\partial x_n^\partial x_1} & \frac{\partial^2 f}{\partial x_n^\partial x_2} 
			& \dots & \frac{\partial^2 f}{\partial x_n^\partial x_n}
		\end{array}\right)
	\end{equation}
	% subsection hesse_matrix (end)
	
	\subsection{Jacobi-Matrix} % (fold)
	\label{sub:jacobi_matrix}
	Jacobi-Matrix $J_f(x)$ zu $f(x)$ = \left( \begin{array}{c}f_1(x)\\\vdots\\f_m(x)\end{array}\right), 
	\mathbb{R}^n \rightarrow \mathbb{R}^m:
	
	\begin{eqnarray}
		\nabla f(x) &=& J_f(x)  = \nabla \left( \begin{array}{c}f_1(x)\\\vdots\\f_m(x)\end{array}\right) 
		= \left( \begin{array}{c} \nabla f_1(x_1,...,x_n) \\ \vdots \\ \nabla f_m(x_1,...,x_n) \end{array} \right) \\
		&=& \left( \begin{array}{ccc} 
				\frac{\partial}{\partial x_1}f_1(x_1,...,x_n) & ... & \frac{\partial}{\partial x_n}f_1(x_1,...,x_n) \\
				... & \ddots & ... \\
				\frac{\partial}{\partial x_1}f_m(x_1,...,x_n) & ... & \frac{\partial}{\partial x_n}f_m(x_1,...,x_n)
			\end{array}\right)
	\end{eqnarray}
	
	% subsection jacobi_matrix (end)
	
	\subsection{Richtungsableitung} % (fold)
	\label{sub:richtungsableitung}
	
	Die Richtungsableitung $D_vf$ einer Funktion $f(x_1,x_2,...,x_n)$ entlang 
	eines Vektors $\overrightarrow{v}$ ($|\overrightarrow{v}| = 1$!) 
	im Punkt $\overrightarrow{p}$ ist definiert als:
	\begin{equation}
		D_vf = \nabla f(p) \cdot \overrightarrow{v}
	\end{equation}
	Das Ergebnis ist ein reiner Zahlenwert, der sich aus dem Skalarprodukt von $\overrightarrow{v}$ mit dem Gradienten-Vektor
	der Funktion $f$ ergibt.
	% subsection richtungsableitung (end)
	
	\subsection{Tangentialabbildung} % (fold)
	\label{sub:tangentialabbildung}
	Sei $f : \mathbb{R}^m \rightarrow \mathbb{R}^n, f(x_1,x_2,...,x_n)$ = y, dann ist $g(x)$ eine affin lineare Abbildung und heißt
	\emph{Tangentialabbildung} von $f$ in \overrightarrow{p}.
	\begin{equation}
		g(x) = f(\overrightarrow{p}) + \nabla f(\overrightarrow{p})\cdot(\overrightarrow{x}-\overrightarrow{p})
	\end{equation}
	Beispiel: An die Funktion $f : \mathbb{R}^2 \rightarrow \mathbb{R}, f(x_1,x_2) = x_1 \cdot x_2$ soll in Punkt $\overrightarrow{p} = 
	\left(\begin{array}{c}1\\1\end{array}\right)$ eine Tangentialebene gelegt werden 
	($\nabla f = \left(\begin{array}{c}x_2\\x_1\end{array}\right)$). 
	\begin{eqnarray*}
		g(x_1,x_2) &=& f(p) + \nabla f(p) \cdot (x-p) \\
		&=& p_1\cdot p_2 + \left(\begin{array}{c}p_2\\p_1\end{array}\right) \cdot
		 \left(\left(\begin{array}{c}x_1\\x_2\end{array}\right) - \left(\begin{array}{c}p_1\\p_2\end{array}\right) \right) \\
		&=& 1 \cdot 1 + \left(\begin{array}{c}1\\1\end{array}\right) \cdot \left(\left(\begin{array}{c}x_1\\x_2\end{array}\right)
		 - \left(\begin{array}{c}1\\1\end{array}\right) \right) \\
		&=& 1 + \left(\begin{array}{c}1\\1\end{array}\right) \cdot \left(\begin{array}{c}x_1 - 1\\x_2 - 1\end{array}\right) \\
		&=& 1+(x_1-1)+(x_2-1) \\
		&=& x_1 + x_2 -1
	\end{eqnarray*}
	% subsection tangentialabbildung (end)
	
	\subsection{Taylorpolynom 2. Grades} % (fold)
	\label{sub:taylorpolynom_2_grades}
	
	Taylorpolynom für $f(x_1,x_2,...,x_n)$ im Punkt $\overrightarrow{p} = (p_1,p_2,...,p_n)$:

	\begin{equation}
		T_2 = f(\overrightarrow{p}) + (\overrightarrow{x}-\overrightarrow{p}) \cdot \nabla f(\overrightarrow{p})+\frac{1}{2}
		(\overrightarrow{x}-\overrightarrow{p})^2 \cdot H_f(\overrightarrow{p})
	\end{equation}
	
	% subsection taylorpolynom_2_grades (end)
	% section differentialrechnung_bei_funktionen_mehrerer_variablen (end)
	
	\pagebreak
	
	\section{Kurven} % (fold)
	\label{sec:kurven}
	
	\subsection{Definition} % (fold)
	\label{sub:definition}
	
	Mehrdimensionale Funktionen der Form $f(t) = (f_1(t),f_2(t),...,f_n(t)), f:[a,b] \rightarrow \mathbb{R}^n$ heißen Kurven.
	Sie haben ein Argument in $\mathbb{R}$ und mehrere Funktionswerte im $\mathbb{R}^n$.
	% subsection definition (end)
	
	\subsection{Länge einer Kurve} % (fold)
	\label{sub:länge_einer_kurve}
	
	Die Länge $L(\gamma)$ einer Kurve $f(t) = (f_1(t),f_2(t),...,f_n(t)), f:[a,b] \rightarrow \mathbb{R}^n$ ist gegeben durch:
	\begin{equation}
		L(\gamma) = \int_a^b \sqrt{f'_1(t)^2 + f'_2(t)^2 + ... + f'_n(t)^2}dt
	\end{equation}
	% subsection länge_einer_kurve (end)
	
	\subsection{Tangentialvektor/Tangente} % (fold)
	\label{sub:tangentialvektor_tangente}
	
	Der Tangentialvektor einer Kurve $\gamma$ im Punkt $t_0$ ist definiert als Einheitsvektor gegeben durch: 
	\begin{equation}
		e(t_0) = \frac{\nabla f(t_0)}{|\nabla f(t_0)|}
	\end{equation}
	Die Tangente $\tau$ an $\gamma$ im Punkt $t_0$ ist demnach:
	\begin{equation}
		\tau : \tau(t) = f(t_0) +  t \cdot e(t_0)
	\end{equation}
	% subsection tangentialvektor (end)
	% section kurven (end)
	
	\pagebreak
	
	\section{Mehrdimensionale Optimierung} % (fold)
	\label{sec:mehrdimensionale_optimierung}
	
	\subsection{Optimierung ohne Nebenbedingungen} % (fold)
	\label{sub:optimierung_ohne_nebenbedingungen}
	Sei $f$ zweimal stetig differenzierbar bei $x_0$.
	\\\\Stationäre Punkte (notwendige Bedingung):
	\begin{equation}
		\nabla f (x_0) = 0
	\end{equation}
	Art des Extremums (hinreichende Bedingung):
	\begin{eqnarray}
	H_f(x_0) = \nabla^2f(x_0) : $\ positiv definit \  \Rightarrow f $ \ ist in \ $ x_0 $ \ lokal minimal\\
	H_f(x_0) = \nabla^2f(x_0) : $\ negativ definit \  \Rightarrow f $ \ ist in \ $ x_0 $ \ lokal maximal\\
	H_f(x_0) = \nabla^2f(x_0) : $\ indefinit \ \Rightarrow f $ \ hat in \ $ x_0 $ \ einen Sattelpunkt
	\end{eqnarray}
	% subsection optimierung_ohne_nebenbedingungen (end)
	
	% section mehrdimensionale_optimierung (end)
	
	\section{Differentialgleichungen} % (fold)
	\label{sec:differentialgleichungen}
	
	\subsection{Homogene lineare DGL 2. Grades mit konstanten Koeffizienten} % (fold)
	%\label{sub:homogene_lineare_dgl_2_grades_mit_konstanten_koeffizienten}
	Homogene lineare Differentialgleichungen 2. Grades haben die Form
	\begin{equation}
		y'' + ay' + by = 0
	\end{equation}
	Sie lassen sich lösen mithilfe der charakteristischen Gleichung
	\begin{equation}
		\lambda^2 + a\lambda + b = 0
	\end{equation}
	Lösung der Gleichung mithilfe der Lösungsformel für quadratische Gleichungen. Anhand des Wertes der Diskriminante $D$ 
	müssen nun 3 Fälle unterschieden werden:
	\begin{itemize}
		
		\item $D > 0$:
			Tritt dieser Fall ein, erhält man 2 Lösungen für $\lambda$.\\
			Die allgemeine Lösung der DGL lautet dann
			\begin{equation}
				y(x) = C_1 e^{\lambda_1 x} + C_2 e^{\lambda_2 x}
			\end{equation} 
		
		\item $D = 0$: 
			In diesem Fall gibt es eine Doppellösung für $\lambda = \lambda_1 = \lambda_2$\\
			Die allgemeine Lösung lautet dann
			\begin{equation}
				y(x) = C_1 e^{\lambda x} + C_2 xe^{\lambda x} = e^{\lambda x}(C_1+C_2 x)
			\end{equation}
		
		\item $D < 0$:
			Die charakteristische Gleichung hat nun die beiden komplexen Lösungen 
			\begin{equation}
				\lambda_{1,2} = -\frac{a}{2} \pm i \cdot \sqrt{\left| \frac{a^2}{4}-b\right|}
			\end{equation}
			mit der Lösungsform
			\begin{equation}
				a \pm ib
			\end{equation}
			($a,b$ stehen in diesem Fall in keiner Verbindung zu den ursprünglichen Koeffizienten $a,b$)\\
			Die Gleichung hat nun die allgemeine Lösung 
			\begin{equation}
				y(x) = C'_1 e^{ax}e^{ibx} + C'_2 e^{ax}e^{-ibx} = e^{ax} \cdot (C'_1 e^{ibx}+C'_2 e^{-ibx})
			\end{equation}
			Mit der Euler'schen Formel $e^{i\varphi} = cos(\varphi)+i \cdot sin(\varphi)$ lässt sich diese Form vereinfachen zu 
			\begin{equation}
				y(x) = e^{ax}(C_1 cos(bx) + C_2 sin(bx))
			\end{equation}
	\end{itemize}
	% subsection homogene_lineare_dgl_2_grades (end)
	% section differentialgleichungen (end)
	
	\pagebreak
	
	\section{Zusammenfassung der wichtigsten Formeln} % (fold)
	\label{sec:zusammenfassung_der_wichtigsten_formeln}
	
	\subsection{Mehrdimensionale Funktionen}
	\label{sub:mehrdimensionale_funktionen}
	\subsubsection{Niveaulinien} 
	\label{ssub:niveaulinien}
	\begin{equation}
		f(x_1,x_2,...,x_n) = y = c
	\end{equation}	
	
	\subsection{Differentialrechnung bei Funktionen mehrerer Variablen}
	\label{sub:differentialrechnung_bei_funktionen_mehrerer_variablen}
	\subsubsection{Richtungsableitung} 
	\label{ssub:richtungsableitung}
	\begin{equation}
		D_vf = \nabla f(p) \cdot \overrightarrow{v}
	\end{equation}	
	\subsubsection{Tangentialabbildung}
	\label{ssub:tangentialabbildung}
	\begin{equation}
		g(x) = f(\overrightarrow{p}) + \nabla f(\overrightarrow{p})\cdot(\overrightarrow{x}-\overrightarrow{p})
	\end{equation}	
	\subsubsection{Taylorpolynom 2. Grades}
	\label{ssub:taylorpolynom_2_grades}
	\begin{equation}
		T_2 = f(\overrightarrow{p}) + (\overrightarrow{x}-\overrightarrow{p}) \cdot \nabla f(\overrightarrow{p})+\frac{1}{2}
		(\overrightarrow{x}-\overrightarrow{p})^2 \cdot H_f(\overrightarrow{p})
	\end{equation}
	\subsection{Kurven} 
	\label{sub:kurven}
	\subsubsection{Länge einer Kurve}
	\label{ssub:länge_einer_kurve}
	\begin{equation}
		L(\gamma) = \int_a^b \sqrt{f'_1(t)^2 + f'_2(t)^2 + ... + f'_n(t)^2}dt
	\end{equation}
	\subsubsection{Tangentialvektor/Tangente} 
	\label{ssub:tangentialvektor_tangente}
	Tangentialvektor
	\begin{equation}
		e(t_0) = \frac{\nabla f(t_0)}{|\nabla f(t_0)|}
	\end{equation}
	Tangente
	\begin{equation}
		\tau : \tau(t) = f(t_0) +  t \cdot e(t_0)
	\end{equation}
	\subsection{Differentialgleichungen}
	\label{sub:differentialgleichungen}
	\subsubsection{Lösung der homogenen linearen DGL 2. Grades für $D>0$}
	\label{ssub:lösung_der_homogenen_linearen_dgl_2_grades_für_d_0}
	\begin{equation}
		y(x) = C_1 e^{\lambda_1 x} + C_2 e^{\lambda_2 x}
	\end{equation}	
	\subsubsection{Lösung der homogenen linearen DGL 2. Grades für $D=0$}
	\label{ssub:lösung_der_homogenen_linearen_dgl_2_grades_für_d_0}
	\begin{equation}
		y(x) = C_1 e^{\lambda x} + C_2 xe^{\lambda x} = e^{\lambda x}(C_1+C_2 x)
	\end{equation}
	\subsubsection{Lösung der homogenen linearen DGL 2. Grades für $D<0$}
	\label{ssub:lösung_der_homogenen_linearen_dgl_2_grades_für_d_0}
		\begin{equation}
			y(x) = C'_1 e^{ax}e^{ibx} + C'_2 e^{ax}e^{-ibx} = e^{ax} \cdot (C'_1 e^{ibx}+C'_2 e^{-ibx})
		\end{equation}
		bzw.
		\begin{equation}
			y(x) = e^{ax}(C_1 cos(bx) + C_2 sin(bx))
		\end{equation}

	
	% section zusammenfassung_der_wichtigsten_formeln (end)
\end{document}